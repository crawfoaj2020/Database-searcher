\documentclass[letterpaper,11pt,twoside,final]{article}
\usepackage{sagian}

\begin{document}

\title{Database Searcher and Log File Converter Design}
\author{Alyssa Crawford}
\date{July 25, 2018}
\coverpage

\section*{Introduction}
The database searcher and log file converter is a stand-alone
application built to faciliate easy retrival of information from instrument
log files, both sqlite database files and .log files. In the case of
database files, the application allows the user to preform basic
searches through a GUI, or to enter SQL statments. These searches can
then be saved for future use. The application can
also convert large folders of log files into an easier to search database.
It was built specifically with the
needs of tech support in mind, however other parties may also find it
useful.

This application uses the Swish Concurrency Engine (\url{https://github.com/becls/swish}) to interface with
the databases, run the conversions, and to display the
results. Electron (\url{https://electronjs.org/}) is used create a framework for displaying the pages generated by Swish.  

\section*{Theory of Operation}

The simulator runs as a Windows service and is configured to restart
automatically 10 seconds after a failure. The failure is logged to the
log file, and an event is logged to the Windows event log.

Like CytoLink, the simulator listens on TCP/IP port 1400 for clients
to send command packets to and receive response packets from the
Gallios instrument. The simulator also listens on TCP/IP port 1401 for
a single client to receive data packets.

Web browsers access the simulator with the HTTP protocol on TCP/IP
port 54321. The web pages use jQuery (\url{http://jquery.com}) and
Twitter Bootstrap (\url{http://getbootstrap.com}).

The simulator's main page allows users to define and load multiple
32-position carousels. Users can add, rename, and delete
carousels. Users can view and select the active carousel and can view
each tube position within the carousel. For each tube position, users
can mark it as empty, use the default FCS file, or upload an FCS/LMD
file. Users can also view and edit the Gallios detector mapping for
each tube.

The main page has a control for viewing and setting the Gallios
configuration to one of: 6 colors, 2 lasers; 8 colors, 2 lasers; and
10 colors, 3 lasers.

The simulator also has a developer page with controls for raising
errors, turning CytoLink on and off, enabling and disabling the
connection timeout, turning the Gallios on and off, setting the
Gallios state, and setting and removing the event limit.




\end{document}
